\documentclass[10pt]{article}
\usepackage{multicol}
\usepackage[margin=2cm,a4paper]{geometry}                % See geometry.pdf to learn the layout options. There are lots.
%\geometry{a4paper}                   % ... or a4paper or a5paper or ... 
\usepackage{graphicx}
\usepackage{amssymb}
\usepackage{epstopdf}
\DeclareGraphicsRule{.tif}{png}{.png}{`convert #1 `dirname #1`/`basename #1 .tif`.png}
\usepackage[utf8]{inputenc}
\usepackage[swedish]{babel}


\title{}
\author{Mikael Svahnberg\\
Blekinge Institute of Technology\\
SE-371 79 Karlskrona SWEDEN\\
Mikael.Svahnberg@bth.se}
%\date{}                                           % Activate to display a given date or no date

\usepackage{array}
\newcolumntype{C}[1]{>{\centering\let\newline\\\arraybackslash\hspace{0pt}}m{#1}}

\newenvironment{goal}[1]
{\subsubsection*{#1}
Efter genomförd kurs skall studenten:
\begin{itemize}
}
{\end{itemize}}


\begin{document}
% \maketitle
\begin{minipage}[t]{\textwidth}
  \centering
  \begin{tabular}{C{\textwidth}}
    {\huge KURSBESKRIVNING}\\
    \hline
    {\small}\\
    {\large Objektorienterad Mjukvarudesign}\\
    {\large Object Oriented Software Design}\\
    {\large Kurs på forskarnivå}\\
    {\large Motsvarar 5 högskolepoäng}\\
    {\small}\\
    \hline
  \end{tabular}
\end{minipage}

\begin{multicols}{2}
\section{Syfte}
Arkitekturen och designen av ett mjukvarusystem påverkar i hög grad
kvaliteten på systemet och kostnaden för utvecklingen. Utgående från
grundläggande objektorienterade begrepp och designprinciper analyseras
och modelleras ett systems struktur och beteende med hjälp av
modelleringsspråket UML (Unified Modelling Language) i en strukturerad
arbetsmetodik, Unified Process.

Arbetsmetodiken utgår från en beskrivning av kundkrav och bygger en
spårbar kedja via olika UML-modeller hela vägen fram till
implementation och testning. Den strukturerade arbetsmetodiken och
modelleringsspråket UML ger ett stöd till designarbetet, men för att
skapa en hållbar programvarudesign krävs också en förståelse för
grundläggande designprinciper och designmönster.

Designmönster är generella lösningsförslag på vanligt förekommande
problem, och som mjukvaruutvecklare förväntas man känna till och kunna
anpassa dessa generella lösningar till de konkreta utmaningar man
försöker lösa. Grundläggande designprinciper beskriver hur man
fördelar olika typer av ansvar mellan klasser för att få löst kopplad
och lättunderhållen programvarukod.
\section{Innehåll}
Kursen omfattar följande:

\begin{itemize}
\item Introduktion till Systemutvecklingsprocessen
\item Introduktion till Kravhantering
\item Tids- och storleksskattningar
\item Planering av arbete
\item Grundläggande begrepp inom objektorienterad modellering
\item Introduktion till modelleringsspråket UML
\item Introduktion till arbetsmetodiken Unified Process
\item Grundläggande designprinciper, t.ex. låg koppling, hög sammanhållning, inkapsling, och polymorfism
\item Introduktion till, och användning av Designmönster
\item Testning, i synnerhet ur mjukvarudesignprocess-perspektivet
\item Introduktion till Mjukvaruarkitekturer
\item Introduktion till Mjukvaruarkitekturstilar
\end{itemize}

\section{Mål}
\begin{goal}{Kunskap och Förståelse}
\item kunna visa kunskap om och förståelse för en systematisk arbetsmetodik
\item kunna visa förståelse för grundläggande principer i objektorienterad programvaruutveckling
\item kunna visa förståelse för UML som modelleringsspråk
\item kunna visa kunskap om grundläggande designprinciper
\item kunna visa kunskap om grundläggande designmönster
\item kunna visa kunskap om grundläggande mjukvaruarkitekturstilar
\item kunna visa förståelse för hur designarbete skapar förutsättningar för testning av mjukvarusystem.
\end{goal}

\begin{goal}{Färdighet och Förmåga}
\item på en grundläggande nivå kunna ta fram och bedöma krav på en programvara
\item på en grundläggande nivå kunna planera och bedöma planer för utvecklingsarbete
\item kunna uttrycka strukturen och beteendet hos ett system i termer av objektorienterade koncept
\item kunna analysera strukturen och beteendet hos ett system i termer av objektorienterade koncept
\item kunna korrekt använda och läsa UML för att uttrycka struktur och beteende hos ett system
\item kunna tillämpa grundläggande designprinciper för en objektorienterad design
\item kunna tillämpa grundläggande designmönster i en objektorienterad design.
\item kunna tillämpa grundläggande arkitekturstilar för ett mjukvarusystem
\item kunna resonera om de kvalitetsegenskaper ett system med en viss arkitekturstil har eller bör ha
\item kunna resonera om och skapa en grundläggande testplan för ett objektorienterat system.
\end{goal}

\begin{goal}{Värderingsförmåga och Förhållningssätt}
\item kunna analysera källkod för eventuella förbättringar
\item kunna analysera och kritiskt diskutera en design för eventuella förbättringar.
\end{goal}

\section{Lärande och Undervisning}
Undervisningen består av ingående muntliga och skriftliga diskussioner. Kursen ges på engelska.

\section{Bedömning och Examination}
Examinationen består av aktivt deltagande i diskussioner samt av ingående bedömningar av olika, av studenter producerade, projekt-, design-, arkitektur- och testrapporter.

Bedömning i kursen är betyget godkänd/underkänd.

\section{Kursvärdering}
Kursansvarig ansvarar för att forskarstudenten har möjlighet att lämna synpunkter på kursen.

\section{Kurslitteratur och övriga lärresurser}
huvudbok:

\begin{itemize}
\item C. Larman, \emph{Applying UML and Patterns}, Prentice Hall, 3rd Edition.
\end{itemize}
  
\noindent referenslitteratur:

\begin{itemize}
\item Gamma, Helm, Johnson, Vlissides, \emph{Design Patterns, Elements of Reusable Object-Oriented Software}, Addison-Wesley Professional.
\item L. Bass, R. Kazman, P. Clements, \emph{Software Architecture in Practice}, Addison-Wesley, 2012.
\item R. Nystrom, \emph{Game Programming Patterns}, Genever Benning, 2014.
\end{itemize}

\section{Kursansvarig}
\makeatletter
\@author
\makeatother

\noindent $\blacksquare$
\end{multicols}
\end{document}


% 1. Syfte
% Syftet är att forskarstudenten skall tillägna sig fördjupade kunskaper och färdigheter i vetenskapsteori.

% 2. Innehåll
% −	Introduktion till vetenskapsteori
% −	Orientering i ämnets frågeställningar
% −	Diskussion av centrala teman i ämnet

% 3. Mål 
% Kunskap och förståelse
% −	fördjupad kunskap i vetenskapsteori

% Färdighet och förmåga
% −	förmåga att självständigt reflektera över och kritiskt diskutera vetenskapsteoretiska problem

% 4. Lärande och undervisning 
% Undervisningen sker i form av föreläsningar och seminariediskussioner. Kursen ges på engelska.  

% 5. Bedömning och examination
% Examination består av aktivt deltagande på föreläsningar och seminarier samt inlämnande och presentation av en skriftlig uppgift.
% Bedömning i kursen är betyget godkänd/underkänd. 

% 6. Kursvärdering 
% Kursansvarig ansvarar för att forskarstudenten har möjlighet att lämna synpunkter på kursen.

% 7. Kurslitteratur och övriga lärresurser 
% −	Smith, A. (2016) On the Philosophy of Science – a comprehensive guide to the history and future of Philosophy of Science. Oxford University Press
% −	Kurskompendium och artiklar

% 8. Kursansvarig
% Thomas Hobbes, Instituionen för vetenskapsteori, filosofi och konstruktivism.


%%% Local Variables:
%%% mode: latex
%%% TeX-master: t
%%% End:
