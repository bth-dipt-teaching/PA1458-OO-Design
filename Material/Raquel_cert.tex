\documentclass{letter}
\usepackage{geometry}                % See geometry.pdf to learn the layout options. There are lots.
\geometry{a4paper}                   % ... or a4paper or a5paper or ... 
\usepackage{graphicx}
\usepackage[latin1]{inputenc}

\usepackage{wallpaper}
\ULCornerWallPaper{0.15}{BTHlogo_big.png}
\usepackage{tablefootnote}

\makeatletter
\let\@texttop\relax
\makeatother

\signature{Mikael Svahnberg\\
Associate Professor\\
Blekinge Institute of Technology\\
Phone: +46 455 385811}

\address{Blekinge Institute of Technology\\
SE-371 79 Karlskrona SWEDEN\\
Mikael.Svahnberg@bth.se}
%\date{}                                           % Activate to display a given date or no date


\longindentation=0pt
\begin{document}
\begin{letter}{\bf To whom it may concern,}
  \opening{}

  During Autumn 2017 Raquel Ouriques (SSN: 820724P780) participated in the courses PA1415 Software Design (7.5 ECTS), PA1435 Object-Oriented Design (6 ECTS), and PA1443 Introduction to Software Design and Architectures (5 ECTS) in the functions of discussion leader and assignment marker. Her contribution included answering questions from the students, leading discussions about object-oriented concepts, assessing object-oriented analyses and designs from students, and providing rich feedback to the students in order to improve their designs.

  Specific topics include:
  \begin{scriptsize}
\begin{itemize}  
\item Introduction to the system development process
\item Introduction to requirements engineering
\item Time- and Size-assessments
\item Work breakdown and planning  
\item Basic concepts in object-oriented modelling
\item Introduction to the Unified Modelling Language (UML)
\item Introduction to the development process Unfied Process
\item Basic Design Principles, such as low coupling, high cohesion, polymorphism
\item Introduction to, and use of Design Patterns
\item Testing, in particular from the perspective of the software design process
\item Introduction to software architectures
\item Introduction to software architecture styles and patterns
\end{itemize}
\end{scriptsize}

The course books used are \emph{C. Larman, Applying UML and Patterns, Prentice Hall, 3rd Edition.} (main book) and \emph{Gamma, Helm, Johnson, Vlissides, Design Patterns, Elements of Reusable Object-Oriented Software, Addison-Wesley Professional.} (reference book).


In order to complete her tasks in the courses Raquel had to study up on the topics and become proficient enough to be able to provide meaningful feedback to the students. The dramatic increase in students receiving grades A and B in the courses this year is a testament to her success.

\closing{Sincerely,}
\end{letter}
\end{document}

%%% Local Variables:
%%% mode: latex
%%% TeX-master: t
%%% End:
